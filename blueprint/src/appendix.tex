\appendix

\chapter{Local sphere eversion}
\label{chap:local_eversion}

The local theory of \Cref{chap:local} is already enough to deduce
Smale's sphere eversion theorem, although it is less natural than going through
the general results of \Cref{chap:global}. The goal of this appendix is to
explain how to do so. In this section $E$ denote a finite dimensional real
vector space equipped with an inner product. Later we will assume it is
3-dimensional. We denote by $𝕊$ the unit sphere in $E$.

Although we want to study immersions of $𝕊$ into $E$, we want to work only with
functions defined on the whole $E$. So we introduce a slightly artificial relation.
We denote by $B$ the open ball with radius $9/10$ around the origin in $E$ and set:

\[
  \Rel := \{(x, y, φ) ∈ J^1(E, E) \;|\; x ∉ B ⇒ \rst{φ}{x^⊥} \text{ is injective}\}.
\]

Of course solutions of this relation restrict to immersions of $𝕊$.

\begin{lemma}
  \label{lem:loc_immersion_rel_open}
  \lean{loc_immersion_rel_open}
  \leanok
  The relation $\Rel$ above is open.
\end{lemma}

\begin{proof}
  \leanok
  The main task is to fix $x_0 \notin B$
  and $\varphi_0 \in L(E, E)$ which is injective on $x_0^\perp$ and prove that,
  for every $x$ close to $x_0$ and $\varphi$ close to $\varphi_0$, $\varphi$ is
  injective on $x^\perp$. This is a typical situation where geometric intuition
  makes it feel like there is nothing to prove.

  One difficulty is that the subspace $x^\perp$ moves with $x$. We reduce to a fixed
  subspace by considering the restriction to $x_0^\perp$ of the orthogonal
  projection onto $x^\perp$. One can check this is an isomorphism as long as $x$
  is not perpendicular to $x_0$.
  More precisely, we consider $f \co J^1(E, E) \to ℝ \times L(x_0^\perp, E)$ which sends
  $(x, y, \varphi)$ to $(\langle x_0, x\rangle, \varphi \circ \pr_{x^\perp} \circ j_0)$
  where $j_0$ is the inclusion of $x_0^\perp$ into $E$. The set $U$ of injective
  linear maps is open in $L(x_0^\perp, E)$ and the map $f$ is continuous
  hence the preimage of $\{0\}^c \times U$ is open. This is good enough for us because
  injectivity of $\varphi \circ \pr_{x^\perp} \circ j_0$ implies injectivity of
  $\varphi$ on the image of $\pr_{x^\perp} \circ j_0$ which is $x^\perp$ whenever
  $\langle  x_0, x\rangle \neq 0$.
\end{proof}

\begin{lemma}
  \label{lem:loc_immersion_rel_ample}
  \lean{loc_immersion_rel_ample}
  \uses{ample_codim_two}
  \leanok
  The relation $\Rel$ above is ample.
\end{lemma}

\begin{proof}
  \leanok
  The core fact here is that if one fixes vector spaces $F$ and $F'$, a dual
  pair $(\pi, v)$ on $F$ and an injective linear map $\varphi \co F \to F'$
  then the updated map $\Upd{p}{\varphi}{w}$ is injective if and only if $w$
  is not in $\varphi(\ker\pi)$. First we assume $\Upd{p}{\varphi}{\varphi(u)}$
  is injective for some $u$ in $\varphi(\ker\pi)$ and derive a contradiction.
  We have $\Upd{p}{\varphi}{\varphi(u)}v = \varphi(u)$ by the general
  definition of updating and also $\Upd{p}{\varphi}{\varphi(u)}u = \varphi(u)$
  since $u$ is in $\ker \pi$. Hence injectivity of $\varphi$ ensure $u = v$,
  which is absurd since $\pi(u) = 0$ and $\pi(v) = 1$. Conversely assume $w$
  is not in $\varphi(\ker\pi)$ and let us prove $\Upd{p}{\varphi}{w}$ is
  injective. Assume $x$ is in the kernel of $\Upd{p}{\varphi}{w}$. Decompose
  $x = u + tv$ with $u \in \ker\pi$ and $t$ a real number. We have
  $\Upd{p}{\varphi}{w}(x) = \varphi(u) + tw$. Hence our assumption on $x$
  implies $t$ vanishes otherwise we would have $w = -t⁻¹\varphi(u)$
  contradicting that $w$ isn't in $\varphi(\ker\pi)$. This vanishing and the
  assumption on $x$ then imply $\varphi(u) = 0$. Since $\varphi$ is injective
  we conclude that $u = 0$ and finally $x = 0$.

  We now turn to $\Rel$. It suffices to prove that for every $\sigma = (x, y,
  \varphi) \in \Rel$ and every dual pair $p = (\pi, v)$ on $E$, the slice
  $\Rel(\sigma, p)$ is ample. If $x$ is in $B$ then $\Rel(\sigma, p)$ is the
  whole $E$ which is obviously ample. So we assume $x$ is not in $B$. Since
  $\sigma$ is in $\Rel$, $\varphi$ is injective on $x^\perp$. The slice is the
  set of $w$ such that $\Upd{p}{\varphi}{w}$ is injective on $x^\perp$. Assume
  first $\ker\pi = x^\perp$. Then $\Upd{p}{\varphi}{w}$ coincides with
  $\varphi$ on $x^\perp$ hence the slice is the whole $E$. Assume now that
  $\ker\pi \neq x^\perp$. The slice is not very easy to picture in this case.
  But one should remember that, up to affine isomorphism, the slice depends
  only on $\ker \pi$. More specifically, if we keep $\pi$ but change $v$ then
  the slice is simply translated in $E$. Here we replace $v$ by the projection
  on $x^\perp$ of the vector dual to $\pi$ rescaled to keep the property
  $\pi(v) = 1$. What has been gained is that we now have $v \in x^\perp$ and
  $x^\perp = (x^\perp \cap \ker \pi) \oplus ℝ v$. Since $\varphi$ is
  injective on $x^\perp$, $\varphi(x^\perp \cap \ker \pi)$ is a hyperplane
  in $x^\perp$ and $\Upd{p}{\varphi}{w}$ is injective on $x^\perp$ if and only
  if $w$ is in the complement of $\varphi(x^\perp \cap \ker \pi)$ according to
  the core fact above. Since it is an hyperplane in $x^\perp$, it has
  codimension at least $2$ in $E$ hence its complement is ample.
\end{proof}

\begin{theorem}[Smale 1958]
  \lean{sphere_eversion_of_loc}
  \label{sphere_eversion_of_loc}
  \leanok
  There is a homotopy of immersion of $𝕊^2$ into $ℝ^3$ from the inclusion map to
	the antipodal map $a \co q ↦ -q$.
\end{theorem}

\begin{proof}
  \leanok
  \uses{lem:loc_immersion_rel_open, lem:loc_immersion_rel_ample, lem:h_principle_open_ample_loc, lem:param_trick}
	We denote by $ι$ the inclusion of $𝕊^2$ into $ℝ^3$.
	We set $j_t = (1-t)ι	+ ta$.
  This is a homotopy from $ι$ to $a$ (but not an immersion for $t=1/2$).
  Using the canonical trivialization of the tangent
	bundle of $ℝ^3$, we can set, for $(q, v) ∈ T𝕊^2$,
	$G_t(q, v) = \mathrm{Rot}_{Oq}^{πt}(v)$, the rotation around axis $Oq$ with
	angle $πt$.
  The family $σ : t ↦ (j_t, G_t)$ is a homotopy of formal immersions
  relating $j^1ι$ to $j^1a$. Those formal solutions are holonomic when $t$ is
  zero or one, so we can reparametrize the family to make such it is holonomic
  when $t$ is close to zero or one. Then we can extend it to a homotopy of
  formal solutions of $\Rel$ using a suitable cut-off ensuring smoothness
  near the orign. The relation $\Rel$ is ample according to
  \Cref{lem:loc_immersion_rel_ample} and then \Cref{lem:param_trick} ensures
  its 1-parameter version $\Rel^ℝ$ is also ample. The relation $\Rel$ is open according to
  \Cref{lem:loc_immersion_rel_open} hence $\Rel^ℝ$ is also ample.
  So we can use \Cref{lem:h_principle_open_ample_loc} to deform our family of
  formal solutions into a holonomic one.
\end{proof}

% vim: set expandtab sw=2 tabstop=2:
